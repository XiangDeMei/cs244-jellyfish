\documentclass{acm_proc_article-sp}

\usepackage{pseudocode}

\begin{document}

\title{Evaluation of Jellyfish topology using Mininet}

\numberofauthors{2} 
\author{
  \alignauthor
    Patrick Costello\\
    \affaddr{Stanford University}\\
    \email{pcostell@stanford.edu}
  \alignauthor
    Lynn Cuthriell\\
    \affaddr{Stanford University}\\
    \email{lcuth@stanford.edu}
}

\maketitle
\begin{abstract}
(will fill out later)
\end{abstract}

% A category with the (minimum) three required fields
\category{Networks}{Network performance evaluation}{Network simulations}

\keywords{Jellyfish, Mininet}

\section{Introduction}

Data center topologies are designed specifically to handle high-bandwidth traffic
between the machines housed together. Special topologies, such as the FatTree 
proposed by Al-Fares et al., handle this traffic better than the standard 
hierarchical model. FatTree topologies are capable of handling the same number of 
hosts for roughly 16 \% of the cost of a hierarchical topology. \cite{fattree}

The FatTree topology, however, does not support incremental scaling. It is impossible
to add a single host; datacenter administrators must upgrade every single
switch in the datacenter to one with more ports. As an example, upgrading a data
center with 8192 nodes requires upgrading all of the existing 32-port switches
to 48-port switches, leading to a 27648 nodes. \cite{jellyfish}


Jellyfish, a topology proposed by Singla et al., is a new network topology
based on random graphs. Random graphs are able to provide high throughput 
because they have a low average path length. \cite{jellyfish}

A Jellyfish topology is designed such that each host is connected to a single 
switch. This switch is connected to all the other switches in the network using 
a regular random graph approximation.

In this paper, we explore the claims made by Singla et al. regarding the improvements
of Jellyfish compared to FatTree topologies. We test per server throughput on 
FatTrees using ECMP and k-shortest paths on Jellyfish. 

\section{Experiment}
\subsection{Mininet}
In order to deploy Jellyfish on Mininet, the first step we took was to write the
random graph construction algorithm given by Singla et al. The algorithm, as
we have implemented, is as follows:
\begin{enumerate}
\item Foreach host, assign in a round robin fashion to switches
\item While we haven't converged, choose two random switches with available ports a, b
  \item Connect a and b
\item Foreach switch s with $>=$ 2 free ports
  \item Choose two random switches a, b
  \item Disconnect a, b
  \item Connect a, s
  \item Connect b, s
\end{enumerate}

\subsection{Openflow}
By default, Mininet does not deploy ECMP or any other routing algorithms that can handle loops. 
We therefore implemented them using Openflow. We used Riplpox to create an external FatTree 
controller for use in Mininet. However, since RiplPox assumes a FatTree topology, for Jellyfish 
we wrote our own controller. 
\subsubsection{ECMP}
ECMP for FatTree is provided by RiplPox, but we needed to write our own ECMP routing module for Jellyfish.
\subsubsection{k Shortest Paths}
We will write a k Shortest Paths routing module for both Jellyfish and FatTree.
\section{Results}
\section{Conclusion}
\section{Current Status}
We've set up and verified the FatTree topology. We've set up the Jellyfish topology, verified it manually, and written an ECMP controller for it. It still has some bugs though.

\bibliographystyle{abbrv}
\bibliography{writeup}

\balancecolumns
\end{document}
